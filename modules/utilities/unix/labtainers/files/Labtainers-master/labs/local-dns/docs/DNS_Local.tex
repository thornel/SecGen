\documentclass[11pt]{article}

\usepackage{times}
\usepackage{epsf}
\usepackage{epsfig}
\usepackage{amsmath, alltt, amssymb, xspace}
\usepackage{wrapfig}
\usepackage{fancyhdr}
\usepackage{url}
\usepackage{verbatim}
\usepackage{fancyvrb}

\usepackage{subfigure}
\usepackage{cite}
\usepackage{hyperref}
\hypersetup{%
    pdfborder = {0 0 0}
}
%\usepackage{cases}
%\usepackage{ltexpprt}
%\usepackage{verbatim}

%\topmargin      -0.70in  % distance to headers
%\headheight     0.2in   % height of header box
%\headsep        0.4in   % distance to top line
%\footskip       0.3in   % distance from bottom line

% Horizontal alignment
\topmargin      -0.50in  % distance to headers
\oddsidemargin  0.0in
\evensidemargin 0.0in
\textwidth      6.5in
\textheight     8.9in 


%\centerfigcaptionstrue

%\def\baselinestretch{0.95}


\newcommand\discuss[1]{\{\textbf{Discuss:} \textit{#1}\}}
%\newcommand\todo[1]{\vspace{0.1in}\{\textbf{Todo:} \textit{#1}\}\vspace{0.1in}}
\newtheorem{problem}{Problem}[section]
%\newtheorem{theorem}{Theorem}
%\newtheorem{fact}{Fact}
\newtheorem{define}{Definition}[section]
%\newtheorem{analysis}{Analysis}
\newcommand\vspacenoindent{\vspace{0.1in} \noindent}

%\newenvironment{proof}{\noindent {\bf Proof}.}{\hspace*{\fill}~\mbox{\rule[0pt]{1.3ex}{1.3ex}}}
%\newcommand\todo[1]{\vspace{0.1in}\{\textbf{Todo:} \textit{#1}\}\vspace{0.1in}}

%\newcommand\reducespace{\vspace{-0.1in}}
% reduce the space between lines
%\def\baselinestretch{0.95}

\newcommand{\fixmefn}[1]{ \footnote{\sf\ \ \fbox{FIXME} #1} }
\newcommand{\todo}[1]{
\vspace{0.1in}
\fbox{\parbox{6in}{TODO: #1}}
\vspace{0.1in}
}

\newcommand{\mybox}[1]{
\vspace{0.2in}
\noindent
\fbox{\parbox{6.5in}{#1}}
\vspace{0.1in}
}


\newcounter{question}
\setcounter{question}{1}

\newcommand{\myquestion} {{\vspace{0.1in} \noindent \bf Question \arabic{question}:} \addtocounter{question}{1} \,}

\newcommand{\myproblem} {{\noindent \bf Problem \arabic{question}:} \addtocounter{question}{1} \,}



\newcommand{\copyrightnotice}[1]{
\vspace{0.1in}
\fbox{\parbox{6in}{\small Copyright \copyright\ 2014\ \ Wenliang Du, Syracuse University.\\
      The development of this document is/was funded by the following grants from
      the US National Science Foundation: No. 1303306 and 1318814.
      This lab was altered and imported into the Labtainer framework by the Naval Postgraduate
      School, Center for Cybersecurity and Cyber Operations under National Science
      Foundation Award No. 1438893.
      Permission is granted to copy, distribute and/or modify this document
      under the terms of the GNU Free Documentation License, Version 1.2
      or any later version published by the Free Software Foundation.
      A copy of the license can be found at http://www.gnu.org/licenses/fdl.html.}}
\vspace{0.1in}
}

\newcommand{\copyrightnoticeA}[1]{
\vspace{0.1in}
\fbox{\parbox{6in}{\small Copyright \copyright\ 2006\ \ Wenliang Du, Syracuse University.\\
      The development of this document was partially funded by
      the National Science Foundation's Course, Curriculum, and Laboratory
      Improvement (CCLI) program under Award No. 0618680 and 0231122.
      This lab was altered and imported into the Labtainer framework by the Naval Postgraduate
      School, Center for Cybersecurity and Cyber Operations under National Science
      Foundation Aware No. 1438893.

      Permission is granted to copy, distribute and/or modify this document
      under the terms of the GNU Free Documentation License, Version 1.2
      or any later version published by the Free Software Foundation.
      A copy of the license can be found at http://www.gnu.org/licenses/fdl.html.}}
\vspace{0.1in}
}


\newcommand{\nocopyrightnotice}[1]{
\vspace{0.1in}
\fbox{\parbox{6in}{\small  
      The development of this document is funded by 
      the National Science Foundation's Course, Curriculum, and Laboratory 
      Improvement (CCLI) program under Award No. 0618680 and 0231122. 
      Permission is granted to copy, distribute and/or modify this document.
      }}
\vspace{0.1in}
}

\newcommand{\idea}[1]{
\vspace{0.1in}
{\sf IDEA:\ \ \fbox{\parbox{5in}{#1}}}
\vspace{0.1in}
}

\newcommand{\questionblock}[1]{
\vspace{0.1in}
\fbox{\parbox{6in}{#1}}
\vspace{0.1in}
}


\newcommand{\minix}{{\tt Minix}\xspace}
\newcommand{\unix}{{\tt Unix}\xspace}
\newcommand{\linux}{{\tt Linux}\xspace}
\newcommand{\ubuntu}{{\tt Ubuntu}\xspace}
\newcommand{\selinux}{{\tt SELinux}\xspace}
\newcommand{\freebsd}{{\tt FreeBSD}\xspace}
\newcommand{\solaris}{{\tt Solaris}\xspace}
\newcommand{\windowsnt}{{\tt Windows NT}\xspace}
\newcommand{\setuid}{{\tt Set-UID}\xspace}
%\newcommand{\smx}{{\tt Smx}\xspace}
\newcommand{\smx}{{\tt Minix}\xspace}
\newcommand{\relay}{{\tt relay}\xspace}
\newcommand{\isys}{{\tt iSYS}\xspace}
\newcommand{\ilan}{{\tt iLAN}\xspace}
\newcommand{\iSYS}{{\tt iSYS}\xspace}
\newcommand{\iLAN}{{\tt iLAN}\xspace}
\newcommand{\iLANs}{{\tt iLAN}s\xspace}
\newcommand{\bochs}{{\tt Bochs}\xspace}

\newcommand\FF{{\mathcal{F}}}

\newcommand{\argmax}[1]{
\begin{minipage}[t]{1.25cm}\parskip-1ex\begin{center}
argmax
#1
\end{center}\end{minipage}
\;
}

\newcommand{\bm}{\boldmath}
\newcommand  {\bx}    {\mbox{\boldmath $x$}}
\newcommand  {\by}    {\mbox{\boldmath $y$}}
\newcommand  {\br}    {\mbox{\boldmath $r$}}


%\pagestyle{fancyplain}
%\lhead[\thepage]{\thesection}      % Note the different brackets!
%\rhead[\thesection]{SEED Laboratories}
%\lfoot[\fancyplain{}{}]{Syracuse University} 
%\cfoot[\fancyplain{}{}]{\thepage} 

\newcommand{\tstamp}{\today}   
%\lhead[\fancyplain{}{\thepage}]         {\fancyplain{}{\rightmark}}
%\chead[\fancyplain{}{}]                 {\fancyplain{}{}}
%\rhead[\fancyplain{}{\rightmark}]       {\fancyplain{}{\thepage}}
%\lfoot[\fancyplain{}{}]                 {\fancyplain{\tstamp}{\tstamp}}
%\cfoot[\fancyplain{\thepage}{}]         {\fancyplain{\thepage}{}}
%\rfoot[\fancyplain{\tstamp} {\tstamp}]  {\fancyplain{}{}}

\pagestyle{fancy}
%\lhead{\bfseries Computer Security Course Project}
\lhead{\bfseries SEED Labs}
\chead{}
\rhead{\small \thepage}
\lfoot{}
\cfoot{}
\rfoot{}




\usepackage{float}
%\documentclass{article} 
%\usepackage{graphicx}
%\usepackage{color}
%\usepackage[latin1]{inputenc}
%\usepackage{lgrind}
%\input {highlight.sty}

\lhead{\bfseries SEED Labs -- Local DNS Attack Lab}


\def \code#1 {\fbox{\scriptsize{\texttt{#1}}}}

\begin{document}

\begin{center}
{\LARGE Local DNS Attack Lab}
\end{center}

\copyrightnoticeA

\section{Lab Overview}

DNS\cite{bib1} (Domain Name System) is the Internet's phone book; it  
translates hostnames to IP addresses (or IP addresses to hostnames).
This translation is through DNS resolution, which happens behind
the scene. DNS Pharming\cite{bib4} attacks manipulate this resolution process
in various ways, with an intent to misdirect users to
alternative destinations, which are often malicious. 
The objective of this lab is to understand how such attacks work.
Students will first set up and configure a DNS server\cite{bib2}, and then they 
will try various DNS Pharming attacks on the target that is also within
the lab environment.


The difficulties of 
attacking local victims versus remote DNS servers are 
quite different. Therefore, we have developed two labs, one focusing 
on local DNS attacks, and the other on remote DNS attack. This lab focuses 
on local attacks.




\section{Lab Environment}
This lab runs in the Labtainer framework,
available at http://my.nps.edu/web/c3o/labtainers.
That site includes links to a pre-built virtual machine
that has Labtainers installed, however Labtainers can
be run on any Linux host that supports Docker containers.

From your labtainer-student directory start the lab using:
\begin{verbatim}
    labtainer local-dns
\end{verbatim}
Links to this lab manual and to an empty lab report will be displayed.  If you create your lab report on a separate system, 
be sure to copy it back to the specified location on your Linux system.

\section{Network Configuration}
The network configuration is illustrated in 
Figure~\ref{fig:dns:environment}. 

\begin{figure}[H]
\centering
\includegraphics*[ width=0.9\textwidth]{Figs/topology.jpg}
\caption{The Lab Environment Setup} 
\label{fig:dns:environment}
\end{figure}


As you can see from Figure~\ref{fig:dns:environment}, 
we set up the DNS server, the user machine and the attacker machine in the same LAN.
We assume that the user machine's IP address is {\tt 192.168.0.100}, the DNS 
Server's IP is {\tt 192.168.0.10} and the attacker machine's IP is {\tt 192.168.0.200}.
Each of these systems reach the Internet via a gatway with IP of {\tt 192.168.0.1}

\paragraph {Note for Instructors:} 
For this lab, a lab session is desirable, especially if students are
not familiar with the tools and the environments. If an instructor
plans to hold a lab session (by himself/herself or by a TA), it
is suggested the following to be covered in the
lab session~\footnote{We assume that the instructor has already covered
the concepts of the attacks in the lecture, so we do not include
them in the lab session.}:
\begin{enumerate}
  \item The use of the Labtainers. 

  \item The use of {\tt Wireshark}, {\tt arpspoof}, and {\tt Netwox} tools.

  \item Configuring the DNS server.
\end{enumerate}

\subsection{Review the DNS server Configuration} 

The {tt BIND 9} server program is installed on the Apollo DNS server\cite{bib3}.
The DNS server reads a configuration file named
{\tt /etc/bind/named.conf} when it starts. This configuration file includes an option 
file, which is called {\tt /etc/bind/named.conf.options}.  Please 
review that file and note this entry:
\begin{verbatim}
options {
       dump-file       "/var/cache/bind/dump.db";
};
\end{verbatim}

\noindent which instructs the DNS server to dump its cache into
a file named: \texttt{/var/cache/bind/dump.db} 
whenever you run the command:
\begin{verbatim}
% sudo rndc dumpdb -cache 	// Dump the cache to dump.db  
\end{verbatim}
\noindent You may delete the cache using:
\begin{verbatim}
% sudo rndc flush         	// Flush the DNS cache
\end{verbatim}


If a change is made to a configuration file, the DNS server must be
restarted:
\begin{verbatim}
% sudo /etc/init.d/bind9 restart
\end{verbatim}


\paragraph{Create zones.}
Assume that we own a domain: {\tt example.com}, which means that we are 
responsible for providing the
definitive answer regarding {\tt example.com}. We direct our DNS server
to provide naming for the example.com domain by defing a \textit{zone}
in a file named {\tt example.conf} in /etc/bind, and then importing
that file into the {\tt named.conf} confguration via an entry in 
the {\tt /etc/bind/named.conf.local} file.
It should be noted that the {\tt example.com}
domain name is reserved for use in documentation, and is not owned
by anybody, so it is safe to use it.
Our zone definition from the example.conf file is below:
\begin{verbatim}
zone "example.com" {
        type master;
        file "/var/cache/bind/example.com.db";
      };

zone "0.168.192.in-addr.arpa" {
        type master;
        file "/var/cache/bind/192.168.0";
      };
\end{verbatim}

Notice how the zone definition refers to the {\tt example.com.db} file,
which defines the DNS resolution for exmple.com and is is referred 
to as a \textit{zone file}.  This file is found in \texttt{/var/cache/bind} directory.
Please refer to RFC 1035 for details of zone file definitions.  Our example.com.db 
definition is repeated below:
\begin{verbatim}
$TTL 3D
@       IN      SOA     ns.example.com. admin.example.com. (
        2008111001      ;serial, today's date + today's serial number
        8H              ;refresh, seconds
        2H              ;retry, seconds
        4W              ;expire, seconds
        1D)             ;minimum, seconds

@       IN      NS      ns.example.com.	;Address of name server
@       IN      MX      10 mail.example.com.	;Primary Mail Exchanger

www     IN      A       192.168.0.101	;Address of www.example.com
mail    IN      A       192.168.0.102	;Address of mail.example.com
ns      IN      A       192.168.0.10	;Address of ns.example.com
*.example.com. IN A     192.168.0.100	;Address for other URL in 
                                          ;example.com. domain
\end{verbatim}

The symbol `@' is a special notation meaning the origin from the {\tt named.conf}. 
Therefore, `@' here stands for {\tt example.com}. `IN' means Internet. 
`SOA' is short for Start Of Authority.
This zone file contains 7 resource records (RRs): a SOA (Start Of Authority) RR, 
a NS (Name Server) RR, a MX (Mail eXchanger) RR, and 4 A (host Address) RRs.

We also need to setup the DNS reverse lookup file.
The directory \texttt{/var/cache/bind/}, includes a reverse DNS lookup file 
called \texttt{192.168.0} for the \texttt{example.com} domain:
\begin{verbatim}
$TTL 3D
@       IN      SOA     ns.example.com. admin.example.com. (
                2008111001
                8H
                2H
                4W
                1D)
@       IN      NS      ns.example.com.

101     IN      PTR     www.example.com.
102     IN      PTR     mail.example.com.
10      IN      PTR     ns.example.com.
\end{verbatim}


\subsection{Revview the User Machine Configuration} 
\label{subsec:user_machine}

On the user machine {\tt 192.168.0.100}, we need to let the machine
192.168.0.10 be the default DNS server. We achieved this by setting
the DNS setting file \texttt{/etc/resolv.conf} of the user machine:

\begin{verbatim}
  nameserver 192.168.0.10 # the ip of the DNS server 
\end{verbatim}


\subsection{Review the Attacker Machine Configuration}

The attacher machine includes Wireshark, Netwox and arpspoof untilities.

\subsection{Expected Output}

After you have reviewed the lab environment, 
test the configuration by issuing the following
command on the user machine: 
\begin{verbatim}
% dig www.example.com
\end{verbatim}

You should be able to see something like this:

\begin{verbatim}
<<>> DiG 9.5.0b2 <<>> www.example.com
;; global options:  printcmd
;; Got answer:
;; ->>HEADER<<- opcode: QUERY, status: NOERROR, id: 27136
;; flags: qr aa rd ra; QUERY: 1, ANSWER: 1, AUTHORITY: 1, ADDITIONAL: 1

;; QUESTION SECTION:
;www.example.com.		IN	A

;; ANSWER SECTION:
www.example.com.	259200	IN	A	192.168.0.101

;; AUTHORITY SECTION:
example.com.		259200	IN	NS	ns.example.com.

;; ADDITIONAL SECTION:
ns.example.com.		259200	IN	A	192.168.0.10

;; Query time: 80 msec
;; SERVER: 192.168.0.10#53(192.168.0.10)
;; WHEN: Tue Nov 11 15:26:32 2008
;; MSG SIZE  rcvd: 82
\end{verbatim}

Note: the {\tt ANSWER SECTION} contains the DNS mapping. Notice that
the IP address of {\tt www.example.com} is now {\tt 192.169.0.101}, which
is what we have set up in the DNS server.
For a simple and clear answer,
we can use {\tt nslookup} instead. 
To do a DNS reverse lookup, issue {\tt dig -x N.N.N.N}.


\section{Lab Tasks}


The main objective of Pharming attacks on a user is to redirect the user
to another machine $B$ when the user tries to get to machine $A$ using
$A$'s host name. For example, when the user tries to access the online banking,
such as {\tt www.chase.com}, if the adversaries can redirect the user 
to a malicious web site that looks very much like the main web site 
of {\tt www.chase.com}, the user might be fooled and give away password
of his/her online banking account.

When a user types in {\tt www.chase.com} in his browsers, the user's machine will issue
a DNS query to find out the IP address of this web site. The attacker's goal
is to fool the user's machine with a faked DNS reply, which resolves
{\tt www.chase.com} to a malicious IP address. There are several ways
to achieve such an attack. In the first task, we will 
use {\tt www.example.com} as the web site that the user wants to access,
instead of using the real web site name {\tt www.chase.com}; 
the {\tt example.com} domain name is reserved for use in 
documentation, and is not owned by anybody. 


\subsection{Task 1: Attackers have already compromised the victim's machine}

\paragraph{Modifying HOSTS file.}
The host name and IP address pairs in the HOSTS file (\texttt{/etc/hosts}) 
are used for local lookup; they take the preference over 
remote DNS lookups. For example, if there is a following 
entry in the HOSTS file in the user's computer, 
the \texttt{www.example.com} will be resolved as \texttt{1.2.3.4} in 
user's computer without asking any DNS server:
\begin{verbatim}
1.2.3.4        www.example.com
\end{verbatim}

\paragraph{Attacks.}
If attackers have compromised a user's machine, they can 
modify the HOSTS file to redirect the user to a malicious site
whenever the user tries to access {\tt www.example.com}. Assume that you have 
already compromised a machine, please try this technique to redirect
{\tt www.example.com} to any IP address that you choose.

Note: \texttt{/etc/hosts} is ignored by the {\tt nslookup} command, 
but will take effect on ping command and web browser etc.

After modifing the /etc/hosts file to test this simple attack, restore it
to its original content, e.g., remove any "example.com" entries.

\subsection{Task 2: Directly Spoof Response to User}

In this attack, the victim's machine has not been compromised, so attackers cannot
directly change the DNS query process on the victim's machine. However,
if attackers are on the same local area network as the victim, they 
can still achieve a great damage. Showed as Figure~\ref{fig:flow1}.

When a user types the name of a web site (a host name, such as {\tt www.example.com})
in a web browser, the user's computer will issue a DNS request to the DNS 
server to resolve the IP address of the host name.  After hearing this DNS 
request, the attackers can spoof a fake DNS response\cite{bib6}. The fake DNS response 
will be accepted by the user's computer if it meets 
the following criteria:

\begin{enumerate}

\item The source IP address must match the IP address of the DNS server.

\item The destination IP address must match the IP address of the user's machine.

\item The source port number (UDP port) must match the port number that the DNS
request was sent to (usually port 53).

\item The destination port number must match the port number that the DNS
request was sent from.

\item The UDP checksum must be correctly calculated. 

\item The transaction ID must match the transaction ID in the DNS request.

\item The domain name in the question section of the reply must match the 
domain name in the question section of the request.

\item The domain name in the answer section must match the domain name in the
question section of the DNS request.

\item The User's computer must receive the attacker's DNS reply before it
receives the legitimate DNS response.
\end{enumerate}


To satisfy the criteria 1 to 8, the attackers can sniff the DNS request message
sent by the victim.  An ARP spoofing attack, such as conducted in the arp-spoof
lab, can be used by the attacker to sniff traffic between the user and the 
DNS server.  Once the attackers can sniff that local traffic,
they can then create a fake DNS response, and send back to the victim,
before the real DNS server does. {\tt Netwox} tool 105 provide a utility to conduct
such sniffing and responding.

Tip: in the {\tt Netwox/Netwag} tool 105, you can use the ``filter'' field to indicate 
the IP address of your target. For example,
in the scenario showing below, you can use {\tt "src host 192.168.0.100"}.
\begin{figure}[H]
\begin{center}
\includegraphics*[ width=0.9\textwidth]{Figs/flow1.jpg}
\end{center}
\caption{Directly Spoof response to user flow} 
\label{fig:flow1}
\end{figure}

Attempt this attack and use Wireshark on the attacker to observe the traffic. Note
that the legitemate DNS reply may arrive at the user's computer prior to the bogus
one generated by Netwox.  This will cause your attack to fail. Think about stratgies
an attacker may employ to slow down responses from the DNS server. 

To complete this task, you are not required to succeed in the attack, but you are
required to save a PCAP file on your attacker home directory named "spoof.pcapng".
This PCAP file should include your spoofed DNS response.
 
\subsection{Task 3: DNS Server Cache Poisoning}

The above attack targets the user's machine. In order to achieve long-lasting
effect, every time the user's machine sends out a DNS query for {\tt www.example.com},
the attacker's machine must send out a spoofed DNS response. 
This might not be so efficient; there is a much better way to conduct attacks 
by targeting the DNS server, instead of the user's machine.


When a DNS server $Apollo$ receives a query, if the host name is not within the 
$Apollo$'s domain,
it will ask other DNS servers to get the host name resolved. Note that in
our lab setup, the domain of our DNS server is {\tt example.com}; therefore,
for the DNS queries of other domains (e.g. {\tt www.google.com}), the DNS server
$Apollo$ will ask other DNS servers.
However, before $Apollo$ asks other DNS servers, it first looks 
for the answer from its own cache; if the answer is there, 
the DNS server $Apollo$ will simply reply with the information from its cache. 
If the answer is not in the cache, the DNS server will 
try to get the answer from other DNS servers. When $Apollo$ gets the 
answer, it will store the answer in the cache, so next time, 
there is no need to ask other DNS servers.

Therefore, if attackers can spoof the response from other DNS 
servers, $Apollo$ will keep the spoofed response in its cache\cite{bib5} for 
certain period of time. Next time, when a user's machine wants to resolve the 
same host name, $Apollo$ will use the spoofed response in the cache 
to reply. This way, attackers only need to spoof once, and 
the impact will last until the cached information expires. 
This attack is called {\em DNS
cache poisoning}.  The following diagram (Figure~\ref{fig:flow2}) illustrates this attack.
\begin{figure}[H]
\centering
\includegraphics*[ width=0.9\textwidth]{Figs/flow2.jpg}
\caption{DNS cache posioning flow} 
\label{fig:flow2}
\end{figure}


We can use the same tool (Netwox 105) for this attack. 
You will also use an ARP spoofing attack, but this time
you want to sniff traffic between the Apollo DNS server and the 
external DNS, which is reached via the gateway.  So set up your
ARP spoofing to sniff traffic between that gateway and the
Apollo DNS server.
Before attacking, 
make sure that the DNS Server's cache is empty. 
You can flush the cache using the following command:
\begin{verbatim}
# sudo rndc flush
\end{verbatim}

The difference between this attack and the previous attack is that 
we are spoofing the response to
DNS server now, so we set the {\tt filter} field to `src host 192.168.0.10', 
which is the IP address of the DNS server.
We also use the {\tt ttl} field (time-to-live) 
to indicate how long we want the fake answer to 
stay in the DNS server's cache.  After the DNS server is poisoned, we can stop 
the {\tt Netwox 105}. If we set {\tt ttl} to 600 (seconds), then DNS server will keep 
giving out the fake answer for the next 10 minutes.

You can tell whether the DNS server is poisoned or not by using the network 
traffic captured by {\tt Wireshark} or by
dumping the DNS server's cache. To dump and view the DNS server's cache, issue
the following command:
\begin{verbatim}
# sudo rndc dumpdb -cache
# sudo cat /var/cache/bind/dump.db
\end{verbatim}


\section{What's Next}


In the DNS cache poisoning attack of this lab, 
we assume that the attacker and the DNS server are on
the same LAN, i.e., the attacker can observe the DNS query message
via ARP spoofing.
When the attacker and the DNS server are not on the same LAN,
the cache poisoning attack becomes much more challenging. If you
are interested in taking on such a challenge, you can 
try our ``Remote DNS Attack Lab''.


\section{Submission}

Students need to submit a detailed lab report to describe what they have done and
what they have observed. Report should include evidence to support 
the observations. Evidence include packet traces, screendumps, etc.
If you edited your lab report on a separate system, copy it back to the Linux system at the location
identified when you started the lab, and do this before running the stoplab command.
After finishing the lab, go to the terminal on your Linux system that was used to start the lab and type:
\begin{verbatim}
    stoplab local-dns
\end{verbatim}
When you stop the lab, the system will display a path to the zipped lab results on your Linux system.  Provide that file to
your instructor, e.g., via the Sakai site.

 
\begin{thebibliography}{10}

\bibitem{bib1}
RFC 1035 Domain Names - Implementation and Specification :
\newblock http://www.rfc-base.org/rfc-1035.html

\bibitem{bib2}
DNS HOWTO :
\newblock http://www.tldp.org/HOWTO/DNS-HOWTO.html

\bibitem{bib3}
BIND 9 Administrator ReferenceManual :
\newblock http://www.bind9.net/manual/bind/9.3.2/Bv9ARM.ch01.html

\bibitem{bib4}
Pharming Guide :
\newblock http://www.technicalinfo.net/papers/Pharming.html
%\newblock http://www.ngssoftware.com/papers/ThePharmingGuide.pdf

\bibitem{bib5}
DNS Cache Poisoning:
\newblock http://www.secureworks.com/resources/articles/other\_articles/dns-cache-poisoning/

\bibitem{bib6}
DNS Client Spoof:
\newblock http://evan.stasis.org/odds/dns-client\_spoofing.txt


\end{thebibliography}
\end{document}
